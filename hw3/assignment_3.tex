% assignment_3.text - Assignment 3 for Data Fusion class (Fall 2014)
% Chanmann Lim - September 2014

\documentclass[a4paper]{article}

\usepackage[margin=1 in]{geometry}
\usepackage{amsmath}
\usepackage{listings}

\everymath{\displaystyle}

\begin{document}
\title{CS 8790: Solution to assignment 3}
\author{Chanmann Lim}
\date{September 24, 2014}
\maketitle

\subsection*{Report:} ~\\
\indent Accessing the calibration of the IDEK's new 3D sensor requires the studies of the expected value and covariance matrix of 
the given 100k measurements in comparison to the ground truth and the error covariance matrix given by
the engineers designing the hardware. \\
\begin{align*}
Ground \; truth = [12.9 \quad 130.4 \quad 23.5]
\end{align*}
\begin{align*}
	Error \; covariance \; matrix (R_{True}) = 
	\begin{bmatrix}
		1 & 1 & 1 \\ 
		1 & 2 & 2 \\
		1 & 2 & 3
	\end{bmatrix}
\end{align*}

In this case, the computed expected value of the measurements is $[12.895992 \quad 130.398454 \quad 23.494780]$ which will
round up to the ground truth. Thus, the sensor is properly calibrated and unbiased. Next, calculating the covariance matrix 
of the measurements which is the expected square error matrix \\
\begin{align*}
R &=
\frac{1}{n-1}\sum_{i=1}^{n}
	\begin{bmatrix}
		\bar{x_i} \\
		\bar{y_i} \\
		\bar{z_i}
	\end{bmatrix}
	\begin{bmatrix}
		\bar{x_i} & \bar{y_i} & \bar{z_i}
	\end{bmatrix} \\
R &=
	\begin{bmatrix}
		0.9974  &  0.9991  &  1.0001 \\
    	0.9991  &  2.0037  &  2.0084 \\
    	1.0001  &  2.0084  &  3.0114
	\end{bmatrix}
\end{align*}

Finally, the expected covariance matrix must be conservative in other words it must be smaller than or equal to the 
true covariance matrix given above.
\begin{align*}
\boxed{Eig(R_{True} - R) >= 0}
\end{align*}
\begin{align*}
Eig(R_{True} - R) = Eig\left(
	\begin{bmatrix}
		1 & 1 & 1 \\ 
		1 & 2 & 2 \\
		1 & 2 & 3
	\end{bmatrix}
	-
	\begin{bmatrix}
		0.9974  &  0.9991  &  1.0001 \\
    	0.9991  &  2.0037  &  2.0084 \\
    	1.0001  &  2.0084  &  3.0114
	\end{bmatrix}
\right)
\end{align*}
\begin{align*}
Eig(R_{True} - R) =
	\begin{bmatrix}
		-0.016873 \\
		0.001233 \\
		0.003055		
	\end{bmatrix}
\end{align*}

Since the eigenvalues of the $(R_{True} - R)$ contain only a relatively small negative value $-0.016873$ 
which might be due to the numerical error in computing eigenvalues, the given covariance matrix is valid.

\subsection*{Code result:}
\begin{align*}
	Expected \; value &= 
	\begin{bmatrix}
		12.895992 & 130.398454 & 23.494780
	\end{bmatrix} \\
	Eigenvalues &= 
	\begin{bmatrix}
		-0.016873 & 0.001233 & 0.003055
	\end{bmatrix}
\end{align*}

\newpage
\subsection*{Appendix:}
\lstinputlisting[language=Matlab, title=\lstname, basicstyle=\footnotesize]{assignment_3.m}

\end{document}